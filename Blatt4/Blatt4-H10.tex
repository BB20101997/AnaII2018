\section*{H10}

Sein $a,b \in \mathbb{R}$, $a < b$, $f : [a,b] \to \mathbb{R}$ und $g:[a,b] \to \mathbb{R}, g(x)=mx$

\subsection*{i)}

\subsubsection*{Behauptung}

	Für $m := \frac{2}{b^{2}-a^{2}} \int\limits_{a}^{b}f(x) \,dx$ gilt:
	
	$\int\limits_{a}^{b}(f(x)-mx)\,dx = 0$
	
\subsubsection*{Beweis}

$\int\limits_{a}^{b}(f(x)-mx)\,dx = 0$

$\Rightarrow \int\limits_{a}^{b}(x)\,dx = \int\limits_{a}^{b}mx\,dx = \frac{b^{2}-a^{2}}{2}m$

$\Rightarrow \frac{2}{b^{2}-a^{2}} \int\limits_{a}^{b}f(x) \,dx = m$

\subsection*{ii)}

\subsubsection*{Behauptung}

Für $m := \frac{3\int\limits_{a}^{b}f(x)x\,dx}{b^{3}-a^{3}}$ gilt:

$\int\limits_{a}^{b}(f(x)-mx)^{2}\,dx$ ist minimal.

\subsubsection*{Beweis}

$ h(m) := \int\limits_{a}^{b}(f(x)-mx)^{2}\,dx$

$ = \int\limits_{a}^{b}(f(x)-mx)(f(x)-mx)\,dx$


$ = \int\limits_{a}^{b}f(x)^{2}-2f(x)mx+m^{2}x^{2}\,dx$

$ = \int\limits_{a}^{b}f(x)^{2}\,dx 
-2m\int\limits_{a}^{b}f(x)x\,dx 
+ m^{2}\int\limits_{a}^{b}x^{2}\,dx$

$ = \int\limits_{a}^{b}f(x)^{2}\,dx 
-2m\int\limits_{a}^{b}f(x)x\,dx 
+ m^{2}\frac{b^{3}-a^{3}}{3}$

$h(m)$ hat ein Extremum für $h^{\prime}(m) = 0$

$ h^{\prime}(m) = 2m\frac{b^{3}-a^{3}}{3} - 2\int\limits_{a}^{b}f(x)x\,dx $

$\Rightarrow 2m\frac{b^{3}-a^{3}}{3} =  2\int\limits_{a}^{b}f(x)x\,dx $

$\Rightarrow m =  \frac{3\int\limits_{a}^{b}f(x)x\,dx}{b^{3}-a^{3}} $

$h^{\prime\prime}(m) = 2\frac{b^{3}-a^{3}}{3} > 0$

$\Rightarrow$ für $m = \frac{3\int\limits_{a}^{b}f(x)x\,dx}{b^{3}-a^{3}}$ ist $h(m)$
ein Minimum.


\subsection*{iii)}

Für $a = 0$ und $b = 1$, sowie $f(x) = 2x^{2}-\frac{x}{2} + \frac{1}{3}$ 

Aus i) folgt:

$m = \frac{2}{b^{2}-a^{2}} \int\limits_{a}^{b}f(x) \,dx $
$m = \frac{2}{1^{2}-0^{2}} \int\limits_{0}^{1}2x^{2}-\frac{x}{2} + \frac{1}{3} \,dx$
$m = 2 \int\limits_{0}^{1}2x^{2}-\frac{x}{2} + \frac{1}{3} \,dx $

$m = 2 \left[\frac{2}{3}x^{3} - \frac{x^{2}}{4} + \frac{x}{3}\right]_{0}^{1} $
$m = 2 (\frac{2}{3} - \frac{1}{4} + \frac{1}{3}) = 2 \frac{3}{4} = 1,5 $




Aus ii) folgt:

$m = \frac{3\int\limits_{a}^{b}f(x)x\,dx}{b^{3}-a^{3}} $
$ = \frac{3\int\limits_{0}^{1}f(x)x\,dx}{1^{3}-0^{3}} $
$ = 3\int\limits_{0}^{1}f(x)x\,dx$
$ = 3\int\limits_{0}^{1}(2x^{2}-\frac{x}{2} + \frac{1}{3})x\,dx  $

$ = 3\int\limits_{0}^{1}2x^{3}-\frac{x^{2}}{2} + \frac{x}{3}\,dx  $
$ = 3 \left[\frac{1}{2}x^{4}-\frac{x^{3}}{6} + \frac{x^{2}}{6}\right]^{1}_{0} $
$ = 3 (\frac{1}{2}-\frac{1}{6} + \frac{1}{6}) = 1,5$


In der Regel ist nicht zu erwarten, dass das $m$ zur Erfüllung von i) mit dem $m$ zur Erfüllung von ii) übereinstimmt, da jeweils nur ein Wert in Frage kommt welche mittels unterschiedlicher Funktionen berechnet werden, welche von den selben Variablen abhängen.