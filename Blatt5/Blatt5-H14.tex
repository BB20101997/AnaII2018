\section*{H14}

\subsection*{i)}

\subsubsection*{Behauptung}

Die Funktion $x \mapsto \frac{x^4-x^3-2x^2-x+5}{(x-1)(x^2+2x+2)}$ kann durch Polynomdivision und Vereinfachung mittels Partialbruchzerlegung in die Form 
$x+\frac{-4x-5}{x^2+2x+2}+\frac{\frac{2}{5}}{(x-1)}+\frac{\frac{2}{5}x-\frac{6}{5}}{x^2+2x+2}$ gebracht werden.

\subsubsection*{Beweis}

$\frac{x^4-x^3-2x^2-x+5}{(x-1)(x^2+2x+2)}$
$= \frac{x^3-2x-3}{x^2+2x+2} + \frac{2}{(x-1)(x^2+2x+2)}$
$= x + \frac{-4x-5}{x^2+2x+2} + \frac{2}{(x-1)(x^2+2x+2)}$

\paragraph*{}

$\frac{2}{(x-1)(x^2+2x+2)}
= \frac{2}{(x-1)(x^3+x^2-2)}
=: \frac{A}{x-1} + \frac{Bx+c}{x^2+2x+2}
= \frac{(A+B)x^2+(2A-B+C)x+(2A-C)}{x^3+x^2-2}$
\paragraph*{}

$\Rightarrow$

1.)$A+B = 0$

2.)$2A-B = 0$

3.)$2A-C = 0$

\paragraph*{}

$\left(\begin{array}{ccc|c}
1 & 1 & 0 & 0 \\ 
2 & -1 & 1 & 0 \\ 
2 & 0 & -1 & 2
\end{array}\right) \to $
$\left(\begin{array}{ccc|c}
1 & 1 & 0 & 0 \\ 
3 & 0 & 1 & 0 \\ 
2 & 0 & -1 & 2
\end{array}\right) \to $
$\left(\begin{array}{ccc|c}
1 & 1 & 0 & 0 \\ 
5 & 0 & 0 & 2 \\ 
2 & 0 & -1 & 2
\end{array}\right)$



$\Rightarrow 5A = 2 \Rightarrow A = \frac{2}{5}$
$ \Rightarrow $

$B = -\frac{2}{5}$

$C = -\frac{6}{5}$

$\Rightarrow x + \frac{-4x-5}{x^2+2x+2} + \frac{2}{(x-1)(x^2+2x+2)}$
$=  x + \frac{-4x-5}{x^2+2x+2} + \frac{\frac{2}{5}}{x-1} - \frac{\frac{2}{5}x+\frac{6}{5}}{x^2+2x+2}$

