\section*{Nr.H3 2-Fach}

\subsection*{a)}

Die definition nach LS setzt belibig oft diff'barkeit Vorraus verwendet,
aber nur die einfache Ableitung im folge Text.
Weiter ist die definition für einen Vorzeichenwechsel recht waage Formuliert.
Auch Konstante Funktionen haben nach unserer Definition Extrempunkte, nach der LS definition
hat z.B. die konstante 1 Funktion kein Extremum.


\subsection*{b)}

\section*{Nr. H3 1-Fach}

\subsection*{a)}

\paragraph*{Vorraussetzung:} Seien $f_{n}$ und $g_{n}$ Funktionsfolgen, $f_{n}, g_{n} : X \rightarrow \mathbb{R}, X \subseteq \mathbb{R}$ wobei auf $X f_{n}$ gleichmäßig gegen die Grenzfunktion $f$ und $g_{n}$ gleichmäßig gegen die Grenzfunktion $g$ Konvergiert.

\paragraph*{Behauptung:} Die Funktionsfolge $(f_{n}+g_{n})$ konvergiert auf X gleichmäßig gegen $(f+g)$.

\paragraph*{Beweis:} Sei $\epsilon \in \mathbb{R}_{>0}$. Dann gibt es $N_{\epsilon f}$ und $N_{\epsilon g}$, so dass gilt: \\
$\forall n \in \mathbb{N} : n \geq N_{\epsilon f} \Rightarrow |f_{n}-f| < \frac{\epsilon} {2}$\\
und \\
$\forall n \in \mathbb{N} : n \geq N_{\epsilon g} \Rightarrow |g_{n}-g| < \frac{\epsilon} {2}$ \\
Sei nun  $N_{\epsilon} := max(N_{\epsilon f}, N_{\epsilon g})$, dann gilt: \\
$\epsilon = \frac{\epsilon} {2}+\frac{\epsilon} {2} > |f_{n}-f|+|g_{n}-g| \geq |f_{n}-f+g_{n}-g| = |(f_{n}+g_{n})-(f+g)|$\\
q.e.d.

\subsection*{b)}


\subsection*{c)}