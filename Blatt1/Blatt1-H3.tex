\section*{Nr.H3 2-Fach}

\subsection*{a)}

Die definition nach LS setzt belibig oft diff'barkeit Vorraus verwendet,
aber nur die einfache Ableitung im folge Text.
Weiter ist die definition für einen Vorzeichenwechsel recht waage Formuliert.
Auch Konstante Funktionen haben nach unserer Definition Extrempunkte, nach der LS definition
hat z.B. die konstante 1 Funktion kein Extremum.


\subsection*{b)}

\section*{Nr. H3 1-Fach}

\paragraph*{Vorraussetzung:} Seien $f_{n}$ und $g_{n}$ Funktionsfolgen, $f_{n}, g_{n} : X \rightarrow \mathbb{R}, X \subseteq \mathbb{R}$ wobei auf $X f_{n}$ gleichmäßig gegen die Grenzfunktion $f$ und $g_{n}$ gleichmäßig gegen die Grenzfunktion $g$ Konvergiert.

\subsection*{a)}

\paragraph*{Behauptung:} Die Funktionsfolge $(f_{n}+g_{n})$ konvergiert auf X gleichmäßig gegen $(f+g)$.

\paragraph*{Beweis:} Sei $\epsilon \in \mathbb{R}_{>0}$. Dann gibt es $N_{\epsilon f}$ und $N_{\epsilon g}$, so dass gilt: \\
$\forall n \in \mathbb{N} :\forall x \in X: n \geq N_{\epsilon f} \Rightarrow |f_{n}(x)-f(x)| < \frac{\epsilon} {2}$\\
und \\
$\forall n \in \mathbb{N} :\forall x \in X: n \geq N_{\epsilon g} \Rightarrow |g_{n}(x)-g(x)| < \frac{\epsilon} {2}$ \\
Sei nun  $N_{\epsilon} := max(N_{\epsilon f}, N_{\epsilon g})$. Sei $n \geq N_{\epsilon}$ und $x \in X$ dann gilt: \\
$\epsilon = \frac{\epsilon} {2}+\frac{\epsilon} {2} > |f_{n}(x)-f(x)|+|g_{n}(x)-g(x)| \geq |f_{n}(x)-f(x)+g_{n}(x)-g(x)| = |(f_{n}(x)+g_{n}(x))-(f(x)+g(x))|$\\
q.e.d.

\subsection*{b)}
\paragraph*{Behauptung:} Für $f$ und $g$ beschränkt auf X gilt: $(f_{n}*g_{n})$ konvergiert auf X gleichmäßig gegen $(f*g)$ 

\paragraph*{Beweis:} Seien $f$ und $g$ auf X beschränkt. Sei $\epsilon \in \mathbb{R}_{>0}$. Da $f_{n}$ und $f_{n}$ gegen beschränkte Funktionen konvergieren existieren $N_{0} \in \mathbb{N}$ und $k \in \mathbb{R}$ mit $\forall n \in \mathbb{N}: \forall x \in X: n \geq N_{0} \Longrightarrow |f_{n}(x)|<k$ und \\
$\forall n \in \mathbb{N}: \forall x \in X: n \geq N_{0} \Longrightarrow |g_{n}(x)|<k$ \\
Da $f_{n}$ gleichmäßig gegen $f$ und $g_{n}$ gleichmäßig gegen $g$ konvergiert, gibt es $N_{\epsilon f}$ und $N_{\epsilon g}$, so dass gilt:\\
$\forall n \in \mathbb{N} : \forall n \in X: n \geq N_{\epsilon f} \Rightarrow |f_{n}-f| < \frac{\epsilon}{2k}$ und \\
$\forall n \in \mathbb{N} : \forall n \in X: n \geq N_{\epsilon g} \Rightarrow |g_{n}-g| < \frac{\epsilon}{2k}$\\
Sein nun $N_{\epsilon} = max(N_{0},N_{\epsilon f},N_{\epsilon g})$.\\
Es folgt für alle $x \in X$ und $n \in \mathbb{N} $ mit $ n \geq N_{\epsilon}$:\\
$\epsilon = \frac{\epsilon}{2} + \frac{\epsilon}{2} = k\frac{\epsilon}{2k} + k\frac{\epsilon}{2k} > k|f_{n}(x)-f(x)|+k|g_{n}(x)-g(x)| > \\
|g_{n}(x)||f_{n}(x)-f(x)|+|f_{n}(x)||g_{n}(x)-g(x)| \geq |g_{n}(x) (f_{n}(x)-f(x))+f_{n}(x)(g_{n}(x)-g(x))| = |f_{n}(x)g_{n}(x)-f(x)g(x)|$
q.e.d.
\subsection*{c)}
\paragraph*{Behauptung:} Die Aussage aus b gilt im Allgemeinen nicht für nichtbeschränkte Funktionen

\paragraph*{Beweis:} Seien $g_{n}(x):=x$ und $f_{n}(x):= \frac{1}{x}+\frac{1}{n}$. Dann konvergiert offensichtlich $g_{n}$ gegen $g$ und $f_{n}$ gegen $f$ mit $g(x)=x$ und $f(x)=\frac{1}{x}$.\\
Dann gilt aber $(f_{n}*g_{n})(x)=1+\frac{x}{n}$ was nur punktweise gegen $(f*g)(x)= 1$ konvergiert.