\section*{Nr.H2}

\subsection*{a)}

\paragraph*{Funktion f Fall 1 $ x \in [0,\pi) $}
$ \Rightarrow \sin(x) \in [0,1) \Rightarrow  \lim\limits_{n \to \infty} sin(x)^{n} = 0 $

\paragraph*{Funktion f Fall 2 $ x = \pi $}
$ \Rightarrow \lim\limits_{n \to \infty} sin(x)^{n} = 1 $

%Für x in [0,\pi) ist sin(x) in [x,1) d.h. sin(x)^n konvergiert gegen 0
%Für x = \pi ist sin(x) = 1 d.h. sin(x)^n konvergiert gegen 1

\paragraph*{Funktion g}
$\forall x \in \mathbb{R} : \lim\limits_{n \to \infty}xe^{-nx^{2}} = 0$
\begin{large}
//TODO 
\end{large}


\subsection*{b)}

\paragraph*{Behauptung} ~\\
$ \sum\limits_{n=1}^\infty \frac{\cos(nx)}{n^{2}} $ ist gleichmäßig konvergent.

\paragraph*{Beweis} ~\\
$ \forall x \in \mathbb{R} : |\frac{\cos(nx)}{n^{2}}| \leq \frac{1}{n^{2}} $
Nach Majorantenkriterium ist die Reihe damit gleichmäßig konvergent.

