\section*{Nr.H2}

\subsection*{a)}
\paragraph*{Vorraussetzung:} Seien die Funktionsfolgen $f_{n}$ und $g_{n}$ wie auf dem Aufgabenblatt definiert. 

\paragraph*{Behauptung:} $f_{n}$ und $g_{n}$ konvergieren Punktweise

\paragraph{Beweis:}

\paragraph*{Funktion f Fall 1 $ x \in [0,\pi) $}
$ \Rightarrow \sin(x) \in [0,1) \Rightarrow  \lim\limits_{n \to \infty} sin(x)^{n} = 0 $

\paragraph*{Funktion f Fall 2 $ x = \pi $}
$ \Rightarrow \lim\limits_{n \to \infty} sin(x)^{n} = 1 $

%Für x in [0,\pi) ist sin(x) in [x,1) d.h. sin(x)^n konvergiert gegen 0
%Für x = \pi ist sin(x) = 1 d.h. sin(x)^n konvergiert gegen 1

\paragraph*{Funktion g}
[z.z. $\forall x \in \mathbb{R} : \lim\limits_{n \to \infty}xe^{-nx^{2}} = 0$] \\
Für den Fall $x = 0$ gilt die Behauptung offensichtlich\\
Sei also $x \neq 0$ und sei $\epsilon \in \mathbb{R}_{>0}$. \\
Sein nun $n \in  \mathbb{N}$ mit $n > -\frac{\ln(\frac{\epsilon}{|x|})}{x^{2}}$
Es gelten die folgenden Äquivalenzen:\\
 $|g_{n}(x)-0|<\epsilon \\
 \Leftrightarrow |x e^{-nx^{2}}|<\epsilon \\
 \Leftrightarrow e^{-nx^{2}}<\frac{\epsilon}{|x|} \\
 \Leftrightarrow -nx^{2}<ln(\frac{\epsilon}{|x|}) \\
 \Leftrightarrow n > -\frac{\ln(\frac{\epsilon}{|x|})}{x^{2}} \\ $
 Hierraus folgt die Behauptung.
 



\subsection*{b)}

\paragraph*{Behauptung} ~\\
$ \sum\limits_{n=1}^\infty \frac{\cos(nx)}{n^{2}} $ ist f.a. $x \in \mathbb{R}$ gleichmäßig konvergent.

\paragraph*{Beweis} ~\\
$ \forall x \in \mathbb{R} : |\frac{\cos(nx)}{n^{2}}| \leq \frac{1}{n^{2}} $
Nach Majorantenkriterium ist die Reihe damit gleichmäßig konvergent.
Und da 

\paragraph*{Behauptung} ~\\
$ \sum\limits_{n=0}^\infty 2^{-nx}$ ist f.a. $x \in \mathbb{R}_{\geq 1}$

