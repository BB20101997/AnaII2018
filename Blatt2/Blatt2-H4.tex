\section*{Nr.H4}

\subsection*{i)}

\subsubsection*{Behauptung}

$f(x)$ ist auf $\mathbb{R} \setminus \{0\}$ beliebig oft diff'bar.
	
Für
	\[ P_{n}(x) = 
		\begin{cases} 
			1                                         &, \text{für } n = 0 \\
			-2x^{3}P_{n-1}(x)-x^{2}P^{\prime}_{n-1}   &, \text{für } n > 0 
		\end{cases} 
	\]
	
gilt $\forall x \in \mathbb{R} \setminus \{0\} \forall n \in \mathbb{N}_{0} : 
	f^{(n)}(x)= P_{n}(\frac{1}{x})exp(- \frac{1}{x^{2}})$.

\subsubsection*{Beweis}

$f^{(n)}(x)$ ist auf $\mathbb{R} \setminus \{0\}$ als Komposition diff'barer Funktionen diffbar.

\paragraph*{IA}
	
Für $n = 0$ gilt $f^{(0)}(x) = exp(- \frac{1}{x^{2}}) = P_{0}(\frac{1}{x})exp(- \frac{1}{x^{2}})$

\paragraph*{IV}

Sein $n \in \mathbb{N}_{0}$ mit $f^{(n)}(x)= P_{n}(\frac{1}{x})exp(- \frac{1}{x^{2}})$

\paragraph*{IS}

$f^{(n+1)}(x) = f^{(n)\prime}(x)\\
	= P_{n}(\frac{1}{x})(-2\frac{1}{x^{3}})exp(- \frac{1}{x^{2}}) 
		+ P^{\prime}_{n}(\frac{1}{x})(-\frac{1}{x^{2}})exp(- \frac{1}{x^{2}}) \\
	= (-2(\frac{1}{x})^{3}P_{n}(\frac{1}{x})
		 -(\frac{1}{x})^{2}P^{\prime}_{n}(\frac{1}{x}))exp(- \frac{1}{x^{2}}) \\
	= P_{n+1}(\frac{1}{x})exp(- \frac{1}{x^{2}})$ 
	
\subsection*{ii) (1-Fach)}

\subsubsection*{Behauptung}

$f$ ist in $x = 0$ beliebig oft diff'bar und $f^{(n)}(0) = 0$

\subsubsection*{Beweis}



%% TODO !!!!!!!!!!!!!!!!!!!!!!!!!!!!!!!!!!!!!!!!!!!!!!!!!!!!!!!!!!!!!!!!!!!!!!!



	