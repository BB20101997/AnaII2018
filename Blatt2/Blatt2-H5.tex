\section*{Nr.H5}

Sei $p$: $\mathbb{R} \to \mathbb{R}$, $p(x)=\sum\limits^{n}_{\nu=0}a_{\nu}x^{\nu}$,
ein (reelles) Polynom $n$-ten Grades, sowie $x_{0} \in \mathbb{R}$.

\subsection*{i)}

\subsubsection*{Behauptung}

$\forall k \in \{1,...,n\}$ $\forall x \in \mathbb{R}$ gilt

\[
	p^{(k)}(x) = \sum\limits^{n-k}_{\nu=0}a_{\nu+k}\frac{(\nu+k)!}{\nu!}(x)^{\nu} 
\]

\subsubsection*{Beweis}

\paragraph*{IA}

Für $k = 1$ gilt:

$
 p^{(1)}(x)
 = \sum\limits^{n}_{\nu=1} a_{\nu} \nu (x)^{\nu-1}
 = \sum\limits^{n-1}_{\nu=0} a_{\nu+1} (\nu+1) (x)^{\nu}
 = \sum\limits^{n-1}_{\nu=0} a_{\nu+1} \frac{(\nu+1)!}{\nu!} (x)^{\nu}
$  


\paragraph*{IV}

Sein $k \in \{1,...,n-1\}$ mit 
$p^{(k)}(x) = \sum\limits^{n-k}_{\nu=0}a_{\nu+k}\frac{(\nu+k)!}{\nu!}x^{\nu} $

\paragraph*{IS}

$ p^{(k+1)} = p^{(k)\prime}$

$ = \sum\limits^{n-k}_{\nu=1} a_{\nu+k} \frac{(\nu+k)!}{\nu!} \nu x^{(\nu-1)}$

$ = \sum\limits^{n-k-1}_{\nu=0} a_{\nu+k+1} \frac{(\nu+k+1)!}{(\nu+1)!} (\nu+1)x^{\nu}$

$ = \sum\limits^{n -(k+1)}_{\nu=0} a_{\nu+(k+1)} \frac{(\nu+(k+1))!}{\nu!} x^{\nu}$

\subsection*{ii)}

\subsection*{iii)}
