\section*{Nr.H5}

Sei $p$: $\mathbb{R} \to \mathbb{R}$, $p(x)=\sum\limits^{n}_{\nu=0}a_{\nu}x^{\nu}$,
ein (reelles) Polynom $n$-ten Grades, sowie $x_{0} \in \mathbb{R}$.

\subsection*{i)}

\subsubsection*{Behauptung}

$\forall k \in \{0,...,n\}$ $\forall x \in \mathbb{R}$ gilt

\[
	p^{(k)}(x) = \sum\limits^{n-k}_{\nu=0}a_{\nu+k}\frac{(\nu+k)!}{\nu!}x^{\nu} 
\]

\subsubsection*{Beweis}

\paragraph*{IA}

Für $k = 0$ gilt:

$
 p^{(0)}(x)
 = \sum\limits^{n}_{\nu=0} a_{\nu} x^{\nu}
 = \sum\limits^{n-0}_{\nu=0} a_{\nu+0} \frac{(\nu+0)!}{\nu!} x^{\nu}
 = \sum\limits^{n-k}_{\nu=0} a_{\nu+k} \frac{(\nu+k)!}{\nu!} x^{\nu}
$  


\paragraph*{IV}

Sein $k \in \{0,...,n-1\}$ mit 
$p^{(k)}(x) = \sum\limits^{n-k}_{\nu=0}a_{\nu+k}\frac{(\nu+k)!}{\nu!}x^{\nu} $

\paragraph*{IS}

$ p^{(k+1)} = p^{(k)\prime}$

$ = \sum\limits^{n-k}_{\nu=1} a_{\nu+k} \frac{(\nu+k)!}{\nu!} \nu x^{(\nu-1)}$

$ = \sum\limits^{n-k-1}_{\nu=0} a_{\nu+k+1} \frac{(\nu+k+1)!}{(\nu+1)!} (\nu+1)x^{\nu}$

$ = \sum\limits^{n -(k+1)}_{\nu=0} a_{\nu+(k+1)} \frac{(\nu+(k+1))!}{\nu!} x^{\nu}$

%% TODO H5 ii) !!!!!!!!!!!!!!!!!!!!!!!!!!!!!!!!!!!!!!!!!!!!!!!!!!!!!!!!!!!!!!!!!!!!!!!!!!!!!

\pagebreak
\subsection*{iii)}

\begin{large}
!Einheiten werden im Folgendem nicht mitgeführt!
\end{large}

Die Geschwindigkeit des Klaviers wird durch $v(t) := h^{\prime}(t) = 2 * c_{2}t + c_{1}$ beschrieben.

Die Beschleunigung des Klaviers wird durch $h^{\prime\prime}(t) = 2 * c_{2}$ beschrieben.

Es folgt:

$c_{2} = \frac{-g}{2} = \frac{9.81}{2} = -4.901 $

$c_{1} = v(t) - 2*c_{2}t \Longrightarrow c_{1} = v(2.5) + 9.81*2.5 = -29.525 + 24.525 = -5$

$c_{0} = h(t) - c_{2}t^{2} - c_{1}t \Longrightarrow c_{0}$

$= h(2.5) + 4.901 * 2.5^{2} + 5 * 2.5 = 7.84375 + 4.901 * 6.25 + 12.5$
 
$ = 7.84375 + 30.63125 + 12.5 = 50.975$

Das zur Frage Stehende Gebäude hat damit eine Höhe von $50.975m$ und das Klavier eine Ausgangs Geschwindigkeit vom $5\frac{m}{s}$ Richtung Boden.
Dies sollte der Höhe von Cap4 entsprechen.
       
 