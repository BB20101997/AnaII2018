\documentclass[12pt,a4paper,oneside,ngerman]{article} %% Dokumenten Parameter und Art des Dokuments
\usepackage[utf8]{inputenc} %% Diese Datei ist im utf8 Format dies ist hier damit Latex uns versteht
\usepackage{amsmath} %% Packet zur verwendung Mathematischer Formeln
\usepackage{amssymb} %% Packet zur verwendung Mathematischer Symbole
\usepackage{microtype} %% Sorgt für besseren umgang mit zu lange/kurzen Zeilen

\title{Ana II Blatt 1}
\author{Bennet Bleßmann, Jonas Lange}

\begin{document}
\maketitle
\section*{Nr.H1}
\subsection*{a)}



\subparagraph*{Behauptung}

$\lim\limits_{x \to \infty}(\cosh(x)^{\frac{1}{x}}) = e$

\subparagraph*{Beweis}

Es gilt: $\cosh(x))^{\frac{1}{x}} = (e^{\ln(\cosh(x))})^{\frac{1}{x}}
= (e^{\frac {\ln(\cosh(x))}{x}*x})^{\frac{1}{x}} = e^{\frac{\ln(\cosh(x)}{x}}$
Es bleibt zu Zeigen, dass $\lim\limits_{x \to \infty}\frac{\ln(\cosh(x))}{x} = 1 $\\ 
Sei $f(x) = \ln(\cos(x))$ und $g(x)=x$ dann gilt: $f'(x) = \frac{e^{x}-e^{-x}}{e^{x}+e^{-x}} $ und $g'(x) = 1$ also  $\lim\limits_{   x \to \infty} \frac{f'(x)}{g'(x)} = 1 $ (nach G2 c) \\
Da ausserderm $g'(x)\neq 0$ und $ \lim\limits_{x\to\infty} |g(x)| = \infty $ folgt die Behauptung nach l'Hopital

\subsection*{b)}

\subparagraph*{Behauptung}

$a \in \mathbb{R}_{+} \lim\limits_{n \to \infty} n(\sqrt[n]{a}-1) = 0$

\subparagraph*{Beweis}

\indent $ n(\sqrt[n]{a}-1) = n\sqrt[n]{a} - n = \sqrt[n]{an^{n}} - n $ \\
\indent $ \Rightarrow \lim\limits_{n \to \infty} n(\sqrt[n]{a}-1) = \lim\limits_{n \to \infty} \sqrt[n]{an^{n}} - n = 0 $



\subsection*{c)}
Sei $ f:\mathbb{R} \to \mathbb{R},f(x) = x + \sin(x)\cos(x) $ und $ g:\mathbb{R} \to \mathbb{R},g(x) = f(x)e^{\sin(x)} $.

\subparagraph*{Behauptung} ~\\
\indent $\lim\limits_{x \to \infty} \dfrac{f'(x)}{g'(x)}$ existiert.\\
\indent $\lim\limits_{x \to \infty} \dfrac{f(x)}{g(x)}$ existiert nicht.

\subparagraph*{Beweis} ~\\
\indent $ f'(x) = \cos(x)^{2}-\sin(x)^{2} = \cos(x)^{2}-(1-\cos(x)^{2}) = 2 cos(x)^{2} - 1 $ \\
\indent $ g'(x) = f'(x)e^{\sin(x)}+f(x)e^{\sin(x)}cos(x)$ \\

%% TODO finish this proof

\section*{Nr.H2}

\subsection*{a)}

\paragraph*{Funktion f Fall 1 $ x \in [0,\pi) $}
$ \Rightarrow \sin(x) \in [0,1) \Rightarrow  \lim\limits_{n \to \infty} sin(x)^{n} = 0 $

\paragraph*{Funktion f Fall 2 $ x = \pi $}
$ \Rightarrow \lim\limits_{n \to \infty} sin(x)^{n} = 1 $

%Für x in [0,\pi) ist sin(x) in [x,1) d.h. sin(x)^n konvergiert gegen 0
%Für x = \pi ist sin(x) = 1 d.h. sin(x)^n konvergiert gegen 1

\paragraph*{Funktion g}
$\forall x \in \mathbb{R} : \lim\limits_{n \to \infty}xe^{-nx^{2}} = 0$
\begin{large}
//TODO 
\end{large}


\subsection*{b)}

\paragraph*{Behauptung} ~\\
$ \sum\limits_{n=1}^\infty \frac{\cos(nx)}{n^{2}} $ ist gleichmäßig konvergent.

\paragraph*{Beweis} ~\\
$ \forall x \in \mathbb{R} : |\frac{\cos(nx)}{n^{2}}| \leq \frac{1}{n^{2}} $
Nach Majorantenkriterium ist die Reihe damit gleichmäßig konvergent.


\section*{Nr.H3 2-Fach}

\subsection*{a)}

Die definition nach LS setzt belibig oft diff'barkeit Vorraus verwendet,
aber nur die einfache Ableitung im folge Text.
Weiter ist die definition für einen Vorzeichenwechsel recht waage Formuliert.
Auch Konstante Funktionen haben nach unserer Definition Extrempunkte, nach der LS definition
hat z.B. die konstante 1 Funktion kein Extremum.


\subsection*{b)}

\section*{Nr. H3 1-Fach}

\subsection*{a)}

\paragraph*{Vorraussetzung:} Seien $f_{n}$ und $g_{n}$ Funktionsfolgen, $f_{n}, g_{n} : X \rightarrow \mathbb{R}, X \subseteq \mathbb{R}$ wobei auf $X f_{n}$ gleichmäßig gegen die Grenzfunktion $f$ und $g_{n}$ gleichmäßig gegen die Grenzfunktion $g$ Konvergiert.

\paragraph*{Behauptung:} Die Funktionsfolge $(f_{n}+g_{n})$ konvergiert auf X gleichmäßig gegen $(f+g)$.

\paragraph*{Beweis:} Sei $\epsilon \in \mathbb{R}_{>0}$. Dann gibt es $N_{\epsilon f}$ und $N_{\epsilon g}$, so dass gilt: \\
$\forall n \in \mathbb{N} : n \geq N_{\epsilon f} \Rightarrow |f_{n}-f| < \frac{\epsilon} {2}$\\
und \\
$\forall n \in \mathbb{N} : n \geq N_{\epsilon g} \Rightarrow |g_{n}-g| < \frac{\epsilon} {2}$ \\
Sei nun  $N_{\epsilon} := max(N_{\epsilon f}, N_{\epsilon g})$, dann gilt: \\
$\epsilon = \frac{\epsilon} {2}+\frac{\epsilon} {2} > |f_{n}-f|+|g_{n}-g| \geq |f_{n}-f+g_{n}-g| = |(f_{n}+g_{n})-(f+g)|$\\
q.e.d.

\subsection*{b)}


\subsection*{c)}
	
\end{document}