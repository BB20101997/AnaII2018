\documentclass[12pt,a4paper,oneside,ngerman]{article}
\usepackage[utf8]{inputenc}
\usepackage{amsmath}

\title{Ana II Blatt 1}
\author{Bennet Bleßmann, Jonas Lange}

\begin{document}
\maketitle
\section*{Nr.H1}
\subsection*{a)}

\subparagraph*{Behauptung}

$\lim\limits_{x \to \infty}(\cosh(x)^{\frac{1}{x}}) = e$

\subparagraph*{Beweis}

Es gilt: $\cosh(x))^{\frac{1}{x}} = (e^{\ln(\cosh(x))})^{\frac{1}{x}}
= (e^{\frac {\ln(\cosh(x))}{x}*x})^{\frac{1}{x}} = e^{\frac{\ln(\cosh(x)}{x}}$

Es bleibt zu Zeigen, dass $\lim\limits_{x \to \infty}\frac{\ln(\cosh(x)}{x} = 1 $\\ 
\indent Es gilt:  


	
\subsection*{b)}

\subparagraph*{Behauptung}

$a \in \rm I \!R_{+} \lim\limits_{x \to \infty} n(\sqrt[n]{a}-1) = ?$


\subparagraph*{Beweis}

$\lim\limits_{x \to \infty } \frac{\ln{\cosh{x}}}{x} $





\subsection*{c)}

\subparagraph*{Behauptung}

\subparagraph*{Beweis}

\section*{Nr.H2}

\subsection*{a)}

\subsection*{b)}


\section*{Nr.H3}

\subsection*{a)}

Die definition nach LS setzt belibig oft diff'barkeit Vorraus verwendet aber nur die einfache Ableitung im folge Text.
Weiter ist die definition für einen Vorzeichenwechsel recht waage Formuliert.
Auch Konstante Funktionen haben nach unserer Definition Extrempunkte, nach der LS definition
hat z.B. die konstante 1 Funktion kein Extremum.


\subsection*{b)}
	
\end{document}
