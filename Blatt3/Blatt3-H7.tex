\section*{H7 2-Fach}

\subsection*{a)}

$Sei f:\mathbb{R}\to\mathbb{R}, f(x) := \cos^{2}(x)$

\subsubsection*{Behauptung}
Für $v \in \mathbb{N}$ gilt:
$f^{(2v-1)} = (-1)^{v} 2^{2v-1} \sin(x)\cos(x) = (-1)^{v} 2^{2v-2} \sin(2x) $ und 
$f^{(2v)} = (-1)^{v} 2^{2v-1}(-2\sin(x)\cos(x)-2\cos(x)\sin(x)) = (-1)^{v} 2^{2v-1}\cos(2x) $

\subsubsection*{Beweis}

Es gilt:

	$\sin(2x) = \sin(x + x) = \sin(x)\cos(x)+\cos(x)\sin(x) = 2\sin(x)\cos(x)$

	$\cos(2x) = \cos(x + x) =\cos(x)\cos(x) - \sin(x)\sin(x) = \cos^{2}(x)-\sin^{2}(x)$

\paragraph*{IA}
Für $v = 1$ gilt: 

$	f^{(2v-1)} = f^{(1)} $

$   = (\cos(x)^{2})^{\prime} $

$	= -2\cos(x)\sin(x) $
	
$	= (-1)^{v} 2^{2v-1} \sin(x)\cos(x) $
	
$	= (-1)^{v} 2^{2v-2} \sin(2x) $

\paragraph*{IV}

Sei $v \in \mathbb{N}$ mit 
$f^{(2v-1)} = (-1)^{v} 2^{2v-1} \sin(x)\cos(x) = (-1)^{v} 2^{2v-2} \sin(2x)$

\paragraph*{IS}

$   f^{(2v)} = (f^{(2v-1)})^{\prime} $

$   = ((-1)^{v} 2^{2v-2} \sin(2x))^{\prime} $

$   = (-1)^{v} 2^{2v-1}\cos(2x) $

$   = (-1)^{v} 2^{2v-1}(\cos^{2}(x)-\sin^{2}(x)) $ \bigskip

$   f^{(2(v+1)-1)} = (f^{(2v)})^{\prime} $

$   = ((-1)^{v} 2^{2v-1}(\cos^{2}(x)-\sin^{2}(x)))^{\prime} $

$   = (-1)^{v} 2^{2v-1}(-2\sin(x)\cos(x)-2\cos(x)\sin(x)) $

$   = (-1)^{v} 2^{2v-1}(-4)\sin(x)\cos(x) $

$   = (-1)^{(v+1)} 2^{2(v+1)-1}\sin(x)\cos(x) $

$   = (-1)^{(v+1)} 2^{2(v+1)-2}\sin(2x) $


\subsection*{b)}